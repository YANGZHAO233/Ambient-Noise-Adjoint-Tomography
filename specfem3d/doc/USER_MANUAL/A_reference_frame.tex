\chapter{Reference Frame Convention}\label{cha:Coordinates}

The code uses the following convention for the Cartesian reference
frame:
\begin{itemize}
\item the $x$ axis points East
\item the $y$ axis points North
\item the $z$ axis points up
\end{itemize}
Note that this convention is different from both the \citet{AkRi80}
convention and the Harvard Centroid-Moment Tensor (CMT) convention.
The Aki \& Richards convention is
\begin{itemize}
\item the $x$ axis points North
\item the $y$ axis points East
\item the $z$ axis points down
\end{itemize}
and the Harvard CMT convention is
\begin{itemize}
\item the $x$ axis points South
\item the $y$ axis points East
\item the $z$ axis points up
\end{itemize}

\subsection*{Source and receiver locations}

The SPECFEM3D Cartesian software code internally uses Cartesian coordinates.
The given locations for sources and receiver locations thus may get
converted. Sources and receiver locations are read in from the \texttt{CMTSOLUTION}
(or \texttt{FORCESOLUTION}) and \texttt{STATIONS} files. Note that
e.g. the \texttt{CMTSOLUTION} and \texttt{FORCESOLUTION} files denotes
the location by \textquotedbl{}longitude/latitude/depth\textquotedbl{}.
We read in longitude as $x$ coordinate, latitude as $y$ coordinate.

In case the flag \texttt{SUPPRESS\_UTM\_PROJECTION} is set to \texttt{.false.}
in the main parameter file (see Chapter~\ref{cha:Creating-Distributed-Databases}),
the $x$/$y$ coordinates have to be given in degrees and are converted
to Cartesian coordinates using a UTM conversion with the specified
UTM zone.

The value for depth (given in $km$ in \texttt{CMTSOLUTION} and \texttt{FORCESOLUTION})
or burial depth (given in $m$ in \texttt{STATIONS}) is evaluated
with respect to the surface of the mesh at the specified $x$/$y$
location to find the corresponding $z$ coordinate. It is possible
to use this depth value directly as $z$ coordinate by changing the
flag \texttt{USE\_SOURCES\_RECVS\_Z} to \texttt{.true.} in the file
\texttt{constants.h} located in the \texttt{src/shared/} subdirectory.


\subsection*{Seismogram outputs}

The seismogram output directions are given in Cartesian $x$/$y$/$z$
directions and not rotated any further. Changing flags in \texttt{constants.h}
in the \texttt{src/shared/} subdirectory only rotates the seismogram
outputs if receivers are forced to be located at the surface (\texttt{RECVS\_CAN\_BE\_BURIED\_EXT\_MESH}
set to \texttt{.false.}) and the normal to the surface at the receiver
location should be used (\texttt{EXT\_MESH\_RECV\_NORMAL} set to \texttt{.true.})
as vertical. In this case, the outputs are rotated to have the vertical
component normal to the surface of the mesh, $x$ and $y$ directions
are somewhat arbitrary orthogonal directions along the surface.

For the labeling of the seismogram channels, see Appendix~\ref{cha:channel-codes}.
Additionally, we add labels to the synthetics using the following
convention: For a receiver station located in an
\begin{description}
\item [{elastic domain:}] ~
\begin{itemize}
\item \texttt{semd} for the displacement vector
\item \texttt{semv} for the velocity vector
\item \texttt{sema} for the acceleration vector
\end{itemize}
\item [{acoustic domain:}] ~\\
 (please note that receiver stations in acoustic domains must be buried.
This is due to the free surface condition which enforces a zero pressure
at the surface.)
\begin{itemize}
\item \texttt{semd} for the displacement vector
\item \texttt{semv} for the velocity vector
\item \texttt{sema} for pressure which will be stored in the vertical component
\texttt{Z}. Note that pressure in the acoustic media is isotropic
and thus the pressure record would be the same in the other two component
directions. We therefore use the other two seismogram components to
store the acoustic potential in component \texttt{X} (or \texttt{N})
and the first time derivative of the acoustic potential in component
\texttt{Y} (or \texttt{E}).
\end{itemize}
\end{description}
The seismograms are by default written out in ASCII-format to the
\texttt{OUTPUT\_FILES/} subdirectory by each parallel process. You
can change this behavior by changing the following flags in the \texttt{constants.h}
file located in the \texttt{src/shared/} subdirectory:
\begin{description}
\item [{\texttt{SEISMOGRAMS\_BINARY}}] set to \texttt{.true.} to have seismograms
written out in binary format.
\item [{\texttt{WRITE\_SEISMOGRAMS\_BY\_MASTER}}] Set to \texttt{.true.}
to have only the master process writing out seismograms. This can
be useful on a cluster, where only the master process node has access
to the output directory.
\item [{\texttt{USE\_OUTPUT\_FILES\_PATH}}] Set to \texttt{.false.} to
have the seismograms output to \texttt{LOCAL\_PATH} directory specified
in the main parameter file \texttt{DATA/Par\_file}.
\end{description}

